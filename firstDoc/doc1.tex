\documentclass[a4paper,12pt]{article}
\usepackage[T1]{fontenc} %police
\usepackage[utf8]{inputenc} % input encoding (permet les acents)
\usepackage[french]{babel}
\usepackage{blindtext}

\usepackage{layout}
\begin{document}
\title{Mon premier doc \LaTeX héhé}
\author{virgile\thanks{coucou1}\\Loria \and thomas\thanks{coucou2}\\ENSEM BOUH}% \and permet de séparer en colonnes
\date{\today}


\maketitle

\begin{abstract}
\blindtext
\end{abstract}

\setcounter{tocdepth}{4} %profondeur de la table des matières
\setcounter{secnumdepth}{3}%profondeur à laquelle on numérote
\setcounter{page}{100}%ùodifie le numéro de page
\tableofcontents
\section{Introduction}

% \newenvironement{testEnv}{\bgroup}{\egroup}
% \begin{testEnv}
%   \blindtext
% \end{testEnv}

\subsection{subsection azeazsdqs}
\blindtext \\
\subsection{subsection wx:ck,ms}
\blindtext \\

\newpage
\section{Listes}
\subsection{listes à puce}

\renewcommand\labelitemi{X}
% \renewcommand\labelitemii{\times}

\begin{itemize}
  \item chat
  \begin{enumerate}
    \item chat 1
    \item chat 2
    \item chat 3
  \end{enumerate}
  \item chien
  \item singes
  \begin{itemize}
    \item babouin
    \item chimpanzée
  \end{itemize}
\end{itemize}

\subsection{environement description}
\begin{description}
  \item[chat] boule de poils
  \item[chien] boule de poils plus grosse \blindtext
\end{description}

\section{Environement non formaté}
\begin{verbatim}
  \begin{itemize}
    \item chat
    \begin{enumerate}
      \item chat 1
      \item chat 2
      \item chat 3
    \end{enumerate}
    \item chien
    \item singes
    \begin{itemize}
      \item babouin
      \item chimpanzée
    \end{itemize}
  \end{itemize}
\end{verbatim}
\newpage
\layout
\end{document}
