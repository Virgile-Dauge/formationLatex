\documentclass[a4paper,12pt]{article}

\usepackage[T1]{fontenc} %police
\usepackage[utf8]{inputenc} % input encoding (permet les acents)
\usepackage[french]{babel}
\usepackage{blindtext} %génération de Lorem Ipsum
\usepackage[reals]{layout} % permet l'affichage du layout d'une page via \layout
\usepackage{fancyhdr}%extended Headers and footers
\begin{document}
\title{Mon premier doc \LaTeX héhé}
\author{virgile\thanks{coucou1}\\Loria \and thomas\thanks{coucou2}\\ENSEM BOUH}% \and permet de séparer en colonnes
\date{\today}

\maketitle
\tableofcontents
%=====================================================================================
%
%  ___       _                        _   _____                    _       _
% |_ _|_ __ | |_ ___ _ __ _ __   __ _| | |_   _|__ _ __ ___  _ __ | | __ _| |_ ___
%  | || '_ \| __/ _ \ '__| '_ \ / _` | |   | |/ _ \ '_ ` _ \| '_ \| |/ _` | __/ _ \
%  | || | | | ||  __/ |  | | | | (_| | |   | |  __/ | | | | | |_) | | (_| | ||  __/
% |___|_| |_|\__\___|_|  |_| |_|\__,_|_|   |_|\___|_| |_| |_| .__/|_|\__,_|\__\___|
%                                                           |_|
% \newpage
% \section{section name}
% \subsection{Usepackage}
% \begin{verbatim}
% \end{verbatim}
% \subsection{How to use}
% \begin{verbatim}
% \end{verbatim}
% \subsection{Results}
%======================================================================================
\newpage
\section{Encoding and language selection}
\subsection{Usepackage}
\begin{verbatim}
\usepackage[T1]{fontenc} % police selection
\usepackage[utf8]{inputenc} % input encoding
\usepackage[french]{babel} % Language selection
\end{verbatim}
\subsection{How to use}
Nothing special to do. Just use your choosed language as usual.
\subsection{Results}
é è ù à ô û ü...
\newpage
\section{Abtract}
\subsection{Usepackage}
No needs for article class.
\subsection{How to use it}
\begin{verbatim}
\begin{abstract}
  Qu’est que c’est?. C’est une phrase français avant le lorem ipsum.
  Lorem ipsum dolor sit amet, consectetuer adipiscing elit. Etiam lobor-
  tis facilisis sem. Nullam nec mi et neque pharetra sollicitudin. Praesent
  imperdiet mi nec ante. Donec ullamcorper, felis non sodales commodo,
  lectus velit ultrices augue, a dignissim nibh lectus placerat pede. Viva-
  mus nunc nunc, molestie ut, ultricies vel, semper in, velit. Ut porttitor.
  Praesent in sapien. Lorem ipsum dolor sit amet, consectetuer adipis-
  cing elit. Duis fringilla tristique neque. Sed interdum libero ut metus.
  Pellentesque placerat. Nam rutrum augue a leo. Morbi sed elit sit amet
  ante lobortis sollicitudin. Praesent blandit blandit mauris. Praesent
  lectus tellus, aliquet aliquam, luctus a, egestas a, turpis. Mauris la-
  cinia lorem sit amet ipsum. Nunc quis urna dictum turpis accumsan
  semper.
\end{abstract}
\end{verbatim}
\subsection{Result}

\begin{abstract}
\blindtext
\end{abstract}

\newpage
\section{Table of content}
\subsection{Usepackage}
No needs for article class.
% \setcounter{tocdepth}{4} %profondeur de la table des matières
% \setcounter{secnumdepth}{3}%profondeur à laquelle on numérote
% \setcounter{page}{100}%ùodifie le numéro de page
\section{Introduction}

% \newenvironement{testEnv}{\bgroup}{\egroup}
% \begin{testEnv}
%   \blindtext
% \end{testEnv}

\subsection{subsection azeazsdqs}
\blindtext \\
\subsection{subsection wx:ck,ms}
\blindtext \\

\newpage
\section{Listes}
\subsection{listes à puce}

\renewcommand\labelitemi{X}
% \renewcommand\labelitemii{\times}

\begin{itemize}
  \item chat
  \begin{enumerate}
    \item chat 1
    \item chat 2
    \item chat 3
  \end{enumerate}
  \item chien
  \item singes
  \begin{itemize}
    \item babouin
    \item chimpanzée
  \end{itemize}
\end{itemize}

\subsection{environement description}
\begin{description}
  \item[chat] boule de poils
  \item[chien] boule de poils plus grosse \blindtext
\end{description}


\newpage
\section{Environement non formaté}
\begin{verbatim}
  \begin{itemize}
    \item chat
    \begin{enumerate}
      \item chat 1
      \item chat 2
      \item chat 3
    \end{enumerate}
    \item chien
    \item singes
    \begin{itemize}
      \item babouin
      \item chimpanzée
    \end{itemize}
  \end{itemize}
\end{verbatim}

\newpage
\section{Extended Headers and footers (fancyhdr)}
\subsection{Usepackage}
\begin{verbatim}
\usepackage{fancyhdr} %extended Headers and footers
\end{verbatim}

\subsection{Simple example}
This is how to create the simple example you can see on this page.
\begin{verbatim}
\pagestyle{fancy}
\lhead{CoucouL}
\chead{CoucouCenter}
\rhead{CoucouR}
\lfoot{}
\cfoot{To: Dean A. Smith}
\rfoot{\thepage}
\renewcommand{\headrulewidth}{0.4pt}
\renewcommand{\footrulewidth}{0.4pt}
\end{verbatim}
\pagestyle{fancy}
\lhead{CoucouL}
\chead{CoucouCenter}
\rhead{CoucouR}
\lfoot{}
\cfoot{To: Dean A. Smith}
\rfoot{\thepage}
\renewcommand{\headrulewidth}{0.4pt}
\renewcommand{\footrulewidth}{0.4pt}

\subsection{Redefining plain style}
\begin{verbatim}
\fancypagestyle{plain}{%
\fancyhf{} % clear all header and footer fields
\fancyfoot[C]{\textbf{\thepage}} % except the center
\renewcommand{\headrulewidth}{0pt}
\renewcommand{\footrulewidth}{0pt}}
\end{verbatim}
% \setlength\oddsidemargin{-5000000sp} décaler la marge
\newpage
\section{layout}
\layout
\end{document}
